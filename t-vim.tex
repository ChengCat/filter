%D \module
%D   [     file=t-vim,
%D      version=2011.12.17,
%D        title=\CONTEXT\ User Module,
%D     subtitle=Vim syntax highlighting,
%D       author=Aditya Mahajan,
%D         date=\currentdate,
%D    copyright=Aditya Mahajan,
%D        email=adityam <at> ieee <dot> org,
%D      license=Simplified BSD License]

\writestatus{loading}{Vim syntax highlighting (ver: 2011.12.17)}

\startmodule [vim]
\usemodule   [filter]           % loads module catcodes
\usemodule   [syntax-highlight] % loads syntax-groups and filter module

\startinterface all
    \setinterfaceconstant {vimrc}           {vimrc}
\stopinterface

\pushcatcodetable   \setcatcodetable\modulecatcodes

\def\vimtyping@id          {vimtyping}
\def\vimtyping@namespace   {@@@@\vimtyping@id}
\def\vimtyping@name        {}

\def\vimrc@id              {vimrc}

\installparameterhandler \vimtyping@namespace \vimtyping@id
\installsetuphandler     \vimtyping@namespace \vimtyping@id

\def\definevimtyping
  {\dodoubleargument\vimtyping@define}

\starttexdefinition vimtyping@define [#1][#2]
    \setupvimtyping[#1][\s!parent=\vimtyping@namespace,#2]

    \edef\vimtyping@name{#1}

    \definesyntaxhighlighting[#1][\s!parent=\vimtyping@namespace#1]

    \setvalue{\e!start raw#1}{\bgroup\obeylines\dodoubleargument\vimtyping@start_raw[#1]}
    \setvalue{\e!stop  raw#1}{\vimtyping@stop_raw}
    \setvalue{inlineraw#1}{\dodoubleargument\vimtyping@inline_raw[#1]}
\stoptexdefinition

\starttexdefinition vimtyping@start_raw [#1][#2]
    % #1 = filter
    % #2 = options
    \egroup %\bgroup in \start#1 

    \edef\vimtyping@name{#1}

    \begingroup % to keep assignments local
    \setupvimtyping[#1][\c!name=,#2]

    \externalfilterparameter\c!before

    \externalfilter@attributes_start \externalfilter@id \c!style \c!color
    \syntaxhighlighting@linenumbering_start
    \processcommacommand[\externalfilterparameter\c!setups]\directsetup
    \gobbleoneargument % For some reason the next argument is 

\stoptexdefinition

\starttexdefinition vimtyping@stop_raw

    \syntaxhighlighting@linenumbering_stop
    \externalfilter@attributes_stop
    \externalfilterparameter\c!after
    \endgroup

\stoptexdefinition

\starttexdefinition vimtyping@inline_raw [#1][#2]
    % #1 = filter
    % #2 = options
    
    \edef\vimtyping@name{#1}

    \begingroup % to keep assignments local
    \setupvimtyping[#1][\c!name=,\c!before=,\c!after=,#2]

    \externalfilter@attributes_start \externalfilter@id \c!style \c!color
    % We assume that the setups set minimal_catcodes
    \processcommacommand[\externalfilterparameter\c!setups]\directsetup

    \vimtyping@inline_raw_indeed
\stoptexdefinition

\starttexdefinition vimtyping@inline_raw_indeed #1
    #1
    \externalfilter@attributes_stop
    \endgroup
\stoptexdefinition

% Mode to testing the dev version of 2context script.
\doifmodeelse{vim-dev}
  {\def\vimtyping@script_name{2context.vim}}
  {\doifmodeelse\s!mkiv
      {\ctxlua{context.setvalue("vimtyping@script_name",resolvers.resolve("full:2context.vim"))}}
      {\def\vimtyping@script_name{kpse:2context.vim}}}

\def\vimtyping@filter_command
  {vim -u \vimrcfilename\space % read global config file
       --noplugin % dont load plugins
       -e % run in ex mode
       -s % silent
       -C % set compatible
       -n % no swap file
       -c "set tabstop=\externalfilterparameter\c!tab" %
       -c "syntax on" %
       -c "set syntax=\externalfilterparameter\c!syntax" %
       -c "let contextstartline=\externalfilterparameter\c!start" %
       -c "let contextstopline=\externalfilterparameter\c!stop" %
       -c "let strip=\getvalue{\vimtyping@id-\c!strip-\externalfilterparameter\c!strip}" %
       -c "let highlight=[\externalfilterparameter\c!highlight]" %
       -c "source \vimtyping@script_name" %
       -c "qa" %
       \externalfilterinputfile\space
       \externalfilteroutputfile}

\setvalue{\vimtyping@id-\c!strip-\v!off}{0}
\setvalue{\vimtyping@id-\c!strip-\v!on}{1}

% Undocumented ... but useful if the user makes a mistake
\setvalue{\vimtyping@id-\c!strip-\v!no}{0}
\setvalue{\vimtyping@id-\c!strip-\v!yes}{1}

\setupvimtyping
  [% \c!tab=4,
   % \c!start=1,
   % \c!stop=0,
   % \c!syntax=context,
   % \c!alternative=pscolor,
   % \c!before=,
   % \c!after=,
   % \c!style=\tttf,
   % \c!color=,
   \c!strip=\v!off, 
   \c!highlight=,
   \c!highlightcolor=lightgray,
   \c!filtercommand=\vimtyping@filter_command,
   % \c!continue=yes,
   % \c!read=\v!yes,
   % \c!readcommand=\syntaxhighlighting@read_command,
   \c!output=\externalfilterbasefile.vimout,
   %\c!setups=syntaxhighlighting@setup,
   \c!filter\c!setups=vimrc@setup,
   % \c!option=\v!packed, % Could be a list
   \s!parent=\syntaxhighlighting@namespace,
   \c!vimrc=,
   % % Numbering options
   % \c!numbering=\v!no,
   % \c!number\c!start=1,
   % \c!number\c!step=1,
   % \c!number\c!continue=\v!no,
   % \c!numberconversion=\v!numbers,
   % \c!number\c!method=\v!first,
   % \c!number\c!location=\v!left,
   % \c!numberstyle=\ttx,
   % \c!numbercolor=,
   % \c!number\c!width=2em,
   % \c!number\c!left=,
   % \c!number\c!right=,
   % \c!number\c!command=,
   % \c!number\c!distance=0.5em,
   % \c!number\c!align=\v!flushright,
  ]

\def\currentvimtyping  {\vimtyping@name}

\defineexternalfilter
  [\vimrc@id]
  [\c!continue=\v!no,
   \c!read=\v!no,
   \c!filtercommand=\empty]

\def\vimrcfilename{NONE}

\startsetups vimrc@setup
  \doifelsenothing{\externalfilterparameter\c!vimrc}
      {\def\vimrcfilename{NONE}}
      {\begingroup
       \expanded{\setupexternalfilter[\vimrc@id][\c!name=\externalfilterparameter\c!vimrc]}

       \edef\externalfilter@name{\vimrc@id}
       \externalfilter@set_filenames
      
       \global\xdef\vimrcfilename{\externalfilter@input_file}
       \endgroup
      }
\stopsetups

\defineframed[vimtodoframed]
             [
               \c!location=\v!low,
               \c!frame=\v!off,
               \c!background=\v!color,
               \c!backgroundcolor=vimtodoyellow,
            ]

\definecolor[vimtodoyellow]
            [h={E0E090}]

\startsetups[vim-minor-groups]
    \definesyntaxgroup
        [SpecialComment]
        [Comment]

    \definesyntaxgroup
        [String,Character,Number,Boolean,Float]
        [Constant]

    \definesyntaxgroup
        [Function]
        [Identifier]

    \definesyntaxgroup
        [Condition,Repeat,Label,Operator,Keyword,Exception]
        [Statement]

    \definesyntaxgroup
        [Include,Define,Macro,PreCondit]
        [Preproc]

    \definesyntaxgroup
        [StorateClass,Structure,Typedef]
        [Type]

    \definesyntaxgroup
        [SpecialChar,Delimiter,Debug]
        [Special]
\stopsetups

\startcolorscheme[pscolor]
    % Vim Preferred groups
    \definesyntaxgroup 
        [Constant]  
        [\c!color={h=007068}]

    \definesyntaxgroup
        [Identifier]
        [\c!color={h=a030a0}]

    \definesyntaxgroup 
        [Statement] 
        [\c!color={h=2060a8}]

    \definesyntaxgroup 
        [PreProc] 
        [\c!color={h=009030}]

    \definesyntaxgroup 
        [Type]      
        [\c!color={h=0850a0}]

    \definesyntaxgroup 
        [Special] 
        [\c!color={h=907000}]

    \definesyntaxgroup 
        [Comment]
        [\c!color={h=606000}]

    \definesyntaxgroup
         [Ignore]

    \definesyntaxgroup
        [Todo]
        [\c!color={h=800000},
         \c!command=\vimtodoframed]


    \definesyntaxgroup 
        [Error] 
        [\c!color={h=c03000}]

    \definesyntaxgroup 
        [Underlined]
        [\c!color={h=6a5acd},
         \c!command=\underbar]

    \setups{vim-minor-groups}

    \definesyntaxgroup
        [Number]
        [\c!color={h=907000}]
\stopcolorscheme

\startcolorscheme[blackandwhite]
    \definesyntaxgroup 
        [Constant]  

    \definesyntaxgroup
        [Identifier]

    \definesyntaxgroup 
        [Statement] 
        [\c!style=bold]

    \definesyntaxgroup 
        [PreProc] 
        [\c!style=bold]

    \definesyntaxgroup 
        [Type]      
        [\c!style=bold]

    \definesyntaxgroup 
        [Special] 

    \definesyntaxgroup 
        [Comment]
        [\c!style=italic]

    \definesyntaxgroup
         [Ignore]

    \definesyntaxgroup 
        [Todo]      
        [\c!command=\inframed]

    \definesyntaxgroup 
        [Error] 
        [\c!command=\overstrike]

    \definesyntaxgroup 
        [Underlined]
        [\c!command=\underbar]

    \setups{vim-minor-groups}

\stopcolorscheme

\startcolorscheme[kate]
    % Temporary definition... will change
    % . kw dsKeyword
    % . dt dsDataType
    % . dv dsDecVal
    % . bn dsBaseN
    % . fl dsFloat
    % . ch dsChar
    % . st dsString
    % . co dsComment
    % . ot dsOthers
    % . al dsAlert
    % . fu dsFunction
    % . re dsRegionMarker
    % . er dsError
    \definesyntaxgroup 
        [kw]  
        [\c!color={h=007020}, \c!style=bold]

    \definesyntaxgroup 
        [dt]  
        [\c!color={h=902000}] 

    \definesyntaxgroup 
        [dv, bn, fl]  
        [\c!color={h=40a070}] 

    \definesyntaxgroup 
        [ch, st]  
        [\c!color={h=4070a0}] 

    \definesyntaxgroup 
        [co]  
        [\c!color={h=60a0b0}, \c!style=italic] 

    \definesyntaxgroup
        [ot]
        [\c!color={h=007020}]

    \definesyntaxgroup
        [al, er]
        [\c!color=red, \c!style=bold]

    \definesyntaxgroup
        [fu]
        [\c!color={h=06287e}]

    \definesyntaxgroup
        [re]

\stopcolorscheme
\popcatcodetable

\stopmodule

