\usemodule[filter]

\defineexternalfilter
  [R]
  [filtercommand={R CMD BATCH -q --save --restore \externalfilterinputfile\space \externalfilteroutputfile},
   output=\externalfilterbasefile.out,
   readcommand=\typefile,
   continue=yes]

\defineexternalfilter
  [Rhidden]
  [filtercommand={R CMD BATCH -q --save --restore \externalfilterinputfile\space \externalfilteroutputfile},
   output=\externalfilterbasefile.out,
   read=no,
   continue=yes]

\starttext
First a test of whether the workspace is persistent:
bla

\startR
a <- "bla"
b <- "blabla"
ls()
\stopR

One R run ends, another begins.

\startR
ls()
\stopR

Now follows a hidden R run which cleans the R workspace

\startRhidden
rm(list=ls())
save.image()
\stopRhidden

What is in the workspace now?

\startR
ls()
\stopR

Then a small test of generating a graphic, in this case a pdf
\startR
ushape <- c(rexp(500000), 12-rexp(500000))
pdf("ushape.pdf")
par(mfrow=c(1,2))
hist(ushape)
plot(density(ushape), main="Density")
dev.off()
\stopR

The graphic \type{ushape.pdf} can be included in the standard \CONTEXT\ way
\startbuffer
\placefigure{An ugly distribution}{\externalfigure[ushape]}
\stopbuffer
\typebuffer
\getbuffer

\startR
x <- rnorm(900)
y <- rexp(900)
# test comment
f <- gl(9,9,900)
summary(aov(y~x+Error(f)))
library(lattice)
pdf("lattice.pdf")
xyplot(y~x|f)
dev.off()
\stopR

With \type{Sweave} lattice graphics calls must be enclosed in
\type{print()} statements but that is not necessary here.

\startbuffer
\placefigure[here]{Lattice graphics}{\externalfigure[lattice]}
\stopbuffer
\typebuffer
\getbuffer

A test string with nasty characters. In R, the result of a statement
is not printed by default. Enclosing the statement in parentheses,
however causes the parser to see only the value of the statement and
applying the \type{print()} method.
\startR
(test <- ".*\\\\ [[{[{]{[{[{}\]\}=?!+%#|<|>@$")
cat(test)
\stopR

A combination
\startbuffer
\placefigure{A combination of two previously used graphics}{
\startcombination[2*1]
 {\externalfigure[ushape][width=.4\textwidth]}{The first graphic, rescaled}
 {\externalfigure[lattice][width=.4\textwidth]}{The second graphic, rescaled}}
\stopcombination
\stopbuffer
\typebuffer
\getbuffer

Testing a function definition.

\startR
a.df <- data.frame(a=1:2, b=rnorm(2))
a.df$a
testfunction <- function(a=NULL, ...) {
 for(i in 1:length(a)) {
   gsub(a[[i]], "([a-r]|[A-R])", "bla")}
 print(a)}
\stopR

What is in the workspace now?

\startR
ls()
\stopR

\stoptext
