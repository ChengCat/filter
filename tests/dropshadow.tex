% A proof of concept implementation of the drops module

% - How do we shift the shadow?

\usemodule[filter]

\traceexternalfilters

\unprotect

\startluacode
  thirddata = thirddata or {}  
  local format = string.format

  -- Unix Only
  local   left_parenthesis = "\\LP"
  local  right_parenthesis = "\\RP"
  local  double_quote      = [["]]
  thirddata.dropshadowsetup = function (settings)
        width, height, radius = tonumber(settings.width), tonumber(settings.height), tonumber(settings.radius)
        print (">>>>>", "filter", width, height, radius)

        local border = 2 -- add x pixels around mask to get reliable corners (can differ with small radii otherwise)
        -- AM: Perhaps inherit from framethickness or shadowthickness?

        -- '-depth 8' is needed for the correct interpretation of the shadow color values
        -- must stand before 'xc:none'; using the '--verbose' parameter showed 16-bit for the first shadow mask otherwise
        local mask_template = [[-size %dx%d -depth 8 xc:none -fill black -stroke none -draw "roundrectangle 1,1 %d,%d %d,%d"]]
        local mask   = format(mask_template, width+border, height+border, width, height, radius, radius)

        -- shifting the shadow in the graphic makes no sense here, as the positioning is done by ConTeXt;
        local shadow_template = [[-background %s -shadow %dx%d+0+0 +repage]]
        local shadow = format(shadow_template, [["black"]], 50, 5)

        local lp, rp, dq = left_parenthesis, right_parenthesis, double_quote
        local gfx_options = [[-units PixelsPerInch -density 72 -quality 00 +set date:create +set date:modify -background "gray(255)" -colorspace rgb ]]

        --  1. create shadow mask
        --  2. create shadow from shadow mask
        --  3. delete mask and save shadow
        local convert = "convert " .. lp .. mask  .. rp 
                                   .. lp .. "+clone " .. shadow   .. rp 
                                   .. " -delete 0 " .. gfx_options 
                                   .. dq .. "\\externalfilteroutputfile" .. dq

        context.setvalue("dropshadowcommand", convert)
        context.setvalue("dropshadowcacheoption", md5.hex(format("%d-%d-%d", width, height, radius)))
    end
\stopluacode

\def\LP{ \letterbackslash( }
\def\RP{ \letterbackslash) }

\defineexternalfilter
    [dropshadow]
    [
      \c!write=\v!no,
      \c!cache\c!option=\dropshadowcacheoption,
      \c!cache=\v!yes,
      \c!output=\externalfilterbasefile.png,
      \c!filtercommand=\dropshadowcommand,
      \c!filter\c!setups=dropshadowsetups,
      \c!width=100pt,
      \c!height=100pt,
      \c!radius=2pt,
      \c!readcommand=\ReadFigure
    ]

\startsetups dropshadowsetups
    \ctxlua{thirddata.dropshadowsetup( {
                  height = "\withoutpt\the\dimexpr\externalfilterparameter\c!height\relax",
                  width  = "\withoutpt\the\dimexpr\externalfilterparameter\c!width\relax",
                  radius = "\withoutpt\the\dimexpr\externalfilterparameter\c!radius\relax",
              })}
\stopsetups

\def\ReadFigure#1{\externalfigure[#1]}

\defineoverlay[dropshadow][{\inlinedropshadow[width=\overlaywidth,height=\overlayheight]}]

\protect

\starttext

\externalfigure[cow][background=dropshadow]

\blank[big]

% Reuse background
\externalfigure[cow][background=dropshadow]

\blank[big]

% Recompute background
\externalfigure[cow][width=3cm, background=dropshadow]

\blank[big]

\framed[width=5cm, height=3cm, foregroundstyle=\bfc, foregroundcolor=white, background=dropshadow]{Hello world}

\stoptext
