%D \module
%D   [     file=t-filter,
%D      version=2010.12.04
%D        title=\CONTEXT\ User Module,
%D     subtitle=Filter,
%D       author=Aditya Mahajan,
%D         date=\currentdate,
%D    copyright=Aditya Mahajan,
%D        email=adityam <at> umich <dot> edu,
%D      license=Simplified BSD License]

\writestatus{loading}{ConTeXt User Module / Filter}

\startmodule    [filter]

\unprotect

%D \section {Initialization}
%D
%D \subsubject {Interface}
%D
%D The first step is to set the interface variables. This allows me to use 
%D \type{\c!filter} etc. in the module definition, and thereby reduces the risk
%D of a typo. Currently, only English names are provided. If someone wants, I
%D can also add other multi-lingual names.

\startinterface all
  \setinterfaceconstant {filter}           {filter}
  \setinterfaceconstant {filtercommand}    {filtercommand}
  \setinterfaceconstant {output}           {output} 
  \setinterfaceconstant {read}             {read} 
  \setinterfaceconstant {readcommand}      {readcommand} 
\stopinterface

\def\m!externalfilter{t-filter}

%D \subsubject {Messages}

\setinterfacemessage{externalfilter}{title}     {\m!externalfilter}
\setinterfacemessage{externalfilter}{notfound}  {file -- cannot be found}
\setinterfacemessage{externalfilter}{missing}   {output file missing}
\setinterfacemessage{externalfilter}{forbidden} {Fatal Error: Cannot use absolute path -- as directory}
\setinterfacemessage{externalfilter}{slash}     {Appending / to directory -- }

%D \subsubject {Name space}
%D
%D I use the name space \type{externalfilter} for all variables. A lot of this
%D code comes from Wolfgang Schuster and is also implemented in
%D \filename{mult-aux.mkiv}. I could just call \type{\installnamespace} macro from
%D \filename{mult-aux.mkiv}. But, that macro is not defined in \MKII; so I
%D repeat simplified versions of the definitions here.

\def\????externalfilter{@@@@externalfilter}
\def\!!externalfilter  {externalfilter}
\def\currentexternalfilter{}
\def\externalfiltercountername{\????externalfilter-\currentexternalfilter-counter}

%D \subsubject {Read value of parameters}

\def\externalfilterparameter#1%
  {\csname
     \docheckparentparameter{\????externalfilter\currentexternalfilter}{#1}%
   \endcsname}

\def\docheckparentparameter#1#2%
  {\ifcsname#1#2\endcsname
      #1#2%
    \else
      \expandafter\redocheckparentparameter\csname#1\s!parent\endcsname{#2}%
    \fi}

\def\redocheckparentparameter#1#2%
  {\ifx#1\relax
      \s!empty
   \else
      \docheckparentparameter{#1}{#2}%
   \fi}

%D \section {Tracing Macros}

\newif\iftraceexternalfilters

\let\traceexternalfilters\traceexternalfilterstrue

\starttexdefinition doshowfilterstate
  \iftraceexternalfilters
    \writestatus\m!externalfilter{current filter : \currentexternalfilter}
    \writestatus\m!externalfilter{base file : \externalfilterbasefile}
    \writestatus\m!externalfilter{input file : \externalfilterinputfile}
    \writestatus\m!externalfilter{output file : \externalfilteroutputfile}
  \fi
\stoptexdefinition

\def\doshowfilterstatus#1%
  {\iftraceexternalfilters
    \writestatus\m!externalfilter{#1}%
  \fi}

\starttexdefinition doshowfiltercommand
    \writestatus\m!externalfilter{command : \externalfilterparameter\c!filtercommand}
\stoptexdefinition

%D \section {The main user macros}
%D
%D \subsubject {Setup values of parameters}

\def\setupexternalfilters
  {\dodoubleargument\dosetupexternalfilters}
  
\def\dosetupexternalfilters[#1][#2]%
  {\ifsecondargument
     \getparameters[\????externalfilter#1][#2]%
   \else
     \getparameters[\????externalfilter][#1]%
   \fi}


%D \subsubject {Define a new filter}

\def\defineexternalfilter
  {\dodoubleargument\dodefineexternalfilter}

\def\dodefineexternalfilter[#1][#2]%
  {\doshowfilterstatus{defining filter : #1}%
   \edef\currentexternalfilter{#1}%
   \getparameters[\????externalfilter#1][\s!parent=\????externalfilter,#2]%
   \doif{\externalfilterparameter\c!continue}\v!yes
      {\expandafter\newcounter\csname\externalfiltercountername\endcsname}%
   \setvalue{\e!start#1}{\bgroup\obeylines\dodoubleargument\dostartexternalfilter[#1]}%
   \setvalue{\e!stop#1}{\doprocessexternalfilter}%
   \setvalue{process#1file}{\dodoubleargument\doprocessexternalfilterfile[#1]}%
   \setvalue{inline#1}{\doinlineexternalfilter[#1]}
   }

\def\dostartexternalfilter[#1][#2]% filter options
  {% Initializations
   \egroup %\bgroup in \start#1 
   \edef\currentexternalfilter   {#1}%
   \begingroup % to keep assignments local
   \getparameters[\????externalfilter#1][\c!name=,#2]%
   \setexternalfilterfilenames
   % Capture the contents of the buffer
   \dostartbuffer[\externalfiltertmpfile][\e!start#1][\e!stop#1]}

\def\doprocessexternalfilterfile[#1][#2]#3%
  {\begingroup
   \edef\currentexternalfilter    {#1}%
   \getparameters[\????externalfilter#1][\c!name=,#2]% TODO: Add continue=yes
   % Currently filters with a pipe (|) fail with continue=yes
   \setexternalfilterdirectory
   \edef\externalfilterinputfile  {#3}%
   \splitfiletype {#3}%
   %BEWARE. \edef doesn not work
   \def\externalfilterbasefile   {\splitoffname}%
   % The output is always in the directory specified by 
   % \c!directory; even if the input is from some other directory
   \def\externalfilteroutputfile{\getexternalfilterdirectory\externalfilterparameter\c!output}%
   \doshowfilterstate
   \doexecuteexternalfilter
   \doreadprocessedfile
   \endgroup}
   
\def\doinlineexternalfilter[#1]%
  {\edef\currentexternalfilter {#1}%
   \begingroup % to keep assignments local
   \getparameters[\????externalfilter#1][\c!name=,\c!continue=\v!no]%
   \setexternalfilterfilenames
   \pushcatcodetable
   \futurelet\next\dodoinlineexternalfilter}

%D \subsubject {Write argument to file verbatim}
%D
%D Surprisingly, there is nothing in the core to define a function that write its
%D argument to a file verbatim. I basically copied the \type{\type} macro.

\def\dodoinlineexternalfilter
  {\ifx\next\bgroup
     \expandafter\dodoinlineexternalfilterA
   \else
     \expandafter\dodoinlineexternalfilterB
   \fi}

\def\dodoinlineexternalfilterA
  {\setcatcodetable \typcatcodesa
   \redoinlineexternalfilter}

\def\dodoinlineexternalfilterB#1%
  {\setcatcodetable \vrbcatcodes
   \def\dododoinlineexternalfilterB##1#1{\redoinlineexternalfilter{##1}}%
   \dododoinlineexternalfilterB}

\newwrite\externalfilterwrite

\def\redoinlineexternalfilter#1%
  {\immediate\openout \externalfilterwrite\externalfilterinputfile
   \immediate\write   \externalfilterwrite{\detokenize{#1}}%
   \immediate\closeout\externalfilterwrite
   \popcatcodetable
   \doexecuteexternalfilter
   \doreadprocessedfile
   \endgroup}



%D \section {Helper Functions}
%D
%D \subsubject {First and last character of a string}

\def\getfirstcharacter#1%
  {\dogetfirstcharacter#1\relax}

\def\dogetfirstcharacter#1#2\relax{#1}

\def\getlastcharacter#1%
  {\@EA\dogetlastcharacter#1\relax}

\def\dogetlastcharacter#1#2%
  {\ifx#2\relax#1\else\@EA\dogetlastcharacter\@EA#2\fi}

%D \subsubject {Set the name of output directory}

\def\setexternalfilterdirectory
  {\edef\getexternalfilterdirectory{\externalfilterparameter\c!directory}%
   \doifsomething{\getexternalfilterdirectory}\dosetexternalfilterdirectory}
   
\def\dosetexternalfilterdirectory
  {\doif{\getfirstcharacter\getexternalfilterdirectory}{/}
      {\writeline
       \showmessage\!!externalfilter{forbidden}\getexternalfilterdirectory
       \batchmode
       \errmessage{}
       \normalend}
   \doifnot{\getlastcharacter\getexternalfilterdirectory}{/}
      {\showmessage\!!externalfilter{slash}\getexternalfilterdirectory
       \edef\getexternalfilterdirectory{\getexternalfilterdirectory/}}}
  


%D \subsubject {Set file names}
%D
%D \type{\externalfilterbasefile} is the name of the temporary file without
%D extension. Its actual value depends on the state of \type{continue} key as
%D well as the value of \type{name} key.

\def\setexternalfilterfilenames
  {\setexternalfilterdirectory
   % Set the name of temp file for the filter
   \doifelse{\externalfilterparameter\c!continue}\v!yes
       {\edef\externalfiltertmpfile{\!!externalfilter-\currentexternalfilter-\csname\externalfiltercountername\endcsname}}
       {\edef\externalfiltertmpfile{\!!externalfilter-\currentexternalfilter}}
   \doifsomething{\externalfilterparameter\c!name}
       {\edef\externalfiltertmpfile{\!!externalfilter-\currentexternalfilter-\externalfilterparameter\c!name}}
   % The following  macros are useful for filter= and filtercommand= options
   % The basename of the external file 
   \edef\externalfilterbasefile  {\jobname-\externalfiltertmpfile}%
   % In MkII, the buffer output is written to \TEXbufferfile{buffername} where
   % the macro \TEXbufferfile is defined as
   %
   %        \def\TEXbufferfile   #1{\bufferprefix#1.\f!temporaryextension}
   %
   % We redefine bufferprefix to include the directory name.
   \doifmode{\s!mkii}
      {\edef\bufferprefix{\getexternalfilterdirectory\jobname-}}
   % In MkIV, we do not need to do such jugglary, because we can specify the
   % name of the file where the buffer has to be saved. This file is
   % \externalfilterinputfile (because it is the input to the filter).
   \edef\externalfilterinputfile {\getexternalfilterdirectory\externalfilterbasefile.\f!temporaryextension}%
   % The name of the file to which the filter output is written
   \edef\externalfilteroutputfile{\getexternalfilterdirectory\externalfilterparameter\c!output}%
   \doshowfilterstate
  }


%D \subsubject {Process Filter}
%D
%D Execute filter, read the output and do book-keeping if needed.

\def\doprocessexternalfilter
  {% By defualt, buffers are in memory in MkIV
   \doifmode\s!mkiv{\savebuffer[\externalfiltertmpfile][\externalfilterinputfile]}%
   % Run external command
   \doexecuteexternalfilter
   \doreadprocessedfile
   \endgroup 
   % Finalization
   \doif{\externalfilterparameter\c!continue}\v!yes
        {\doglobal\expandafter\increment\csname\externalfiltercountername\endcsname}%
   }

%D \subsubject {Execute Filter}   

\def\doexecuteexternalfilter
  {\doshowfiltercommand
   \doifelse{\externalfilterparameter\c!continue}\v!yes
     {\doifmode{*first}
       {\executesystemcommand
        {mtxrun --ifchanged=\externalfilterinputfile\space 
            --direct \externalfilterparameter\c!filtercommand}}}
     {\executesystemcommand
        {\externalfilterparameter\c!filtercommand}}}

%D \subsubject {Read output}

\def\doreadprocessedfile
  {\doif{\externalfilterparameter\c!read}\v!yes
      {\doiffileelse{\externalfilteroutputfile}
          {\dodoreadprocessedfile}
          {\showmessage\!!externalfilter{notfound}\externalfilteroutputfile 
           \blank
           {\tttf [[\getmessage\!!externalfilter{missing}]]}%
           \blank
          }}}

\def\dodoreadprocessedfile
  {\begingroup
   \externalfilterparameter\c!before
   \processcommacommand[\externalfilterparameter\c!setups]\directsetup
   \externalfilterparameter\c!readcommand\externalfilteroutputfile
   \externalfilterparameter\c!after
   \endgroup}

%D \section {Default Values}

\setupexternalfilters
  [
   \c!before=,
   \c!after=,
   \c!setups=,
   \c!continue=\v!no,
   \c!read=\v!yes,
   \c!readcommand=\ReadFile,
   \c!directory=,
   \c!output=\externalfilterbasefile.tex,
   \c!filter=,
   \c!filtercommand={\externalfilterparameter\c!filter\space \externalfilterinputfile},
 ]
   
\protect
\stopmodule
